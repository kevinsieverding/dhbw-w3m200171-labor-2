% !TeX root = ../index.tex

\section{Setup}

The setup for this lab consists of three virtual machines.
Two running Ubuntu 16.04 and one running Ubuntu 22.04.

\begin{table}[H]
  \centering
  \begin{tabular}{|l|l|l|}
    \hline
    \textbf{Host} & \textbf{IP}    & \textbf{OS}  \\
    \hline
    sieverding-0  & 141.72.188.55  & Ubuntu 22.04 \\
    \hline
    sieverding-1  & 141.72.188.143 & Ubuntu 16.04 \\
    \hline
    sieverding-2  & 141.72.188.140 & Ubuntu 16.04 \\
    \hline
  \end{tabular}
  \caption{Overview of virtual machines}
\end{table}

\section{Getting Software on Your System}\label{sec:installation}

In this exercise we install the open source version of Tripwire---a security tool that can be used to compare the state of a file system against a base line.
For the purpose of later exercises, we do this on machine 2 (sieverding-1).

\begin{enumerate}
  \item Install the required dependencies.

        \begin{minted}{text}
          $ sudo apt update
          $ sudo apt install build-essential git
        \end{minted}

  \item Clone the source of Tripwire from GitHub.

        \begin{minted}{text}
          $ git clone https://github.com/Tripwire/tripwire-open-source
          $ cd tripwire-open-source
        \end{minted}

  \item Build the project according to the instructions in \texttt{README.md}.

        \begin{enumerate}
          \item Execute \texttt{touchconfig.sh} to prevent errors around autoconf/automake.

                \begin{minted}{text}
                  $ ./touchconfig.sh
                \end{minted}

          \item Build the project.

                \begin{minted}{text}
                  $ CXXFLAGS="$CXXFLAGS -std=c++14" ./configure
                  $ make
                \end{minted}

                In my setup, I had to specify that the C++ compiler should use C++ 14 as the standard, since otherwise it would try to use C++ 17 which caused problems.

          \item The \texttt{bin} directory should now contain the built binaries.

          \item Install the binaries to your system.

                \begin{minted}{text}
                  $ sudo make install
                \end{minted}

                In the process of installing you are asked to read and accept the license agreement.
                The setup also asks for passwords to sign the program's database and configuration files.
                This is important to prevent attackers from hiding tampering on the file system from the scanner by modifying these files.

          \item Verify that the installation was successful by calling Tripwire.

                \begin{minted}{text}
                  $ sudo tripwire --version
                \end{minted}

                This should print some version information of the tripwire build.

        \end{enumerate}
\end{enumerate}

Building software from source can be considered more secure than using pre-built binaries, since binaries downloaded from the internet may have malicious alterations.
When building from source, the source code can be inspected and the build result should be pure.
However this is not guaranteed, since the build tools themselves may be compromised and inject malicious code into the build result.

One way to ensure that the build of a piece of software has not been tampered with, is by employing cryptography.
The provider of a piece of software may also provide pre-built binaries.
To give the user the ability to verify that the binaries have not been tampered with, the provider may calculate a cryptographically secure checksum (e.~g. SHA-256) for the binary and provide it alongside it.
This way, the user can download the binary, calculate the checksum themselves and compare it to the one from the provider.
Thereby, they can ensure that the binary has not been modified.

The package management systems of most Linux distributions like apt use checksumming for this reason.
The repositories not only contain software distributions but also checksums and the clients automatically compute checksums for downloaded software packages and compare them against the ones from the repository.
\parencite{debian_wiki_secureapt_nodate}

Checksumming of software distributions safeguards against tampering in transport.
However, the user still has to trust the provider that they did not include malicious code in the binary in the first place.

\section{First Steps With Tripwire}

\subsection{Configure Tripwire}

After we have installed Tripwire, we can now configure and use it.
The \texttt{make install} command has already created a default policy in \enquote{/usr/local/etc/twpol.txt} which covers all of the most important file locations already.
Among others, Tripwire's own binaries and configuration files, global configuration files in \enquote{/etc}, OS and user installed binaries and libraries and the system's boot files.

It also includes \enquote{/home} where the system's user's home directories reside.
While this is a point where malicious software is likely to first enter the system, these directories should also see a large amount of regular activity.
Therefore, we may decide to exclude home directories from our scans, to not have to deal with too much noise when evaluating against our snapshot later.

Let us exclude \enquote{/home} from our scans by changing the policy file as follows:

\begin{minted}{diff}
  (
    rulename = "Monitor Filesystems",
  )
  {
    /                             -> $(SEC_READONLY) ;
  -  /home                         -> $(SEC_READONLY) ;  # Modify as needed
  +  #/home                         -> $(SEC_READONLY) ;
    /usr                          -> $(SEC_READONLY) ;
    /var                          -> $(SEC_READONLY) ;
  }
\end{minted}

Then update the policy for Tripwire.

\begin{minted}{text}
  $ sudo tripwire --update-policy /usr/local/etc/twpol.txt
\end{minted}

The command will update the policy, re-scan the file system and update the database.
If it detects changes to the file system since the last scan it will fail.
This can be prevented by adding the option \texttt{--secure-mode low}.

\subsection{Simulate and Detect an Attack}

Let's simulate an attack where the attacker has somehow injected malicious code to the system program \enquote{mv}.

\begin{itemize}
  \item Create a backup of the \enquote{mv} command.

        \begin{minted}{text}
          $ cp /usr/bin/mv .
          $ mv mv mv-backup
        \end{minted}

  \item Simulate injecting malicious code by appending some text to the binary.

        \begin{minted}{text}
          $ sudo -iu root
          $ echo "println all-your-data" >> /usr/bin/mv
          $ exit
        \end{minted}

        You can check that the binary has actually been changed by manually computing and comparing the SHA-256 checksums of the backup and the changed binary.

        \begin{minted}{text}
          $ sha256sum mv-backup
          $ sha256sum /usr/bin/mv
        \end{minted}

  \item Check the file system with Tripwire.

        \begin{minted}{text}
          $ sudo tripwire --check \
            | tee tripwire-report.txt \ # for later reference
            | less                      # for easier reading
        \end{minted}

        The report shows that the binary was modified.

        \begin{minted}{text}
          ------------------------------------------------------
          Rule Name: OS Binaries and Libraries (/usr/bin)
          Severity Level: 0
          ------------------------------------------------------

          Added:
          "/usr/bin/mv"

          Modified:
          "/usr/bin"
        \end{minted}

\end{itemize}

For Tripwire to be effective it is paramount that its configuration files and data base are protected against tampering.
The most basic way to do this is to ensure the correct access permissions on these files.
Furthermore, Tripwire supports cryptographically signing these files to be able to detect any tampering.
This process is built into the install script.

\section{An Actual Attack}

In this exercise we install Tripwire on a third machine and infect it with an actual rootkit.

\begin{enumerate}
  \item Install Tripwire as described in \ref{sec:installation}.
  \item Initialize the database.

        \begin{minted}{text}
          $ sudo tripwire --init
        \end{minted}

  \item Copy the rootkit to the onto the virtual machine.

        \begin{minted}{text}
          $ rsync -av lilly sieverding-2:/home/ubuntu/.ssh/.lilly
        \end{minted}

  \item Build the rootkit.

        \begin{minted}{text}
          $ cd /home/ubuntu/.ssh/.lilly
          $ make
        \end{minted}

  \item Infect the machine.

        \begin{minted}{text}
          $ sudo insmod lilyofthevalley.ko
        \end{minted}

  \item Build the client to use the rootkit.

        \begin{minted}{text}
          $ make client
        \end{minted}

  \item You can then use the client to elevate yourself to root.

        \begin{minted}{text}
          # Without any options this will drop you into a root shell
          $ ./client
        \end{minted}

\end{enumerate}

Once the rootkit is installed it hides itself from other programs by hiding all files that start with \enquote{lilyofthevalley}.
The most likely way to detect the rootkit would be to find the installation files and/or the client in the user's home directory.
The installation files may be deleted after the system has been infected, but an attacker probably wants to keep the client for easy use of the rootkit.
This client could be hidden by moving it to a location on the filesystem that is hidden by the rootkit.

\begin{enumerate}
  \item Open a root shell.

        \begin{minted}{text}
          $ /home/ubuntu/.ssh/.lilly/client
        \end{minted}

  \item Move the client to a system location so that it is not accidentally deleted by a user and rename it so that it is hidden.

        \begin{minted}{text}
          $ mv /home/ubuntu/.ssh/.lilly/client /etc/lilyofthevalley-client
        \end{minted}

  \item Cover your tracks.

        \begin{minted}{text}
          $ rm -rf /home/ubuntu/.ssh/.lilly
          $ exit # exit the root shell
        \end{minted}

  \item Use the client, because you know where it is.

        \begin{minted}{text}
          $ /etc/lilyofthevalley-client
        \end{minted}

\end{enumerate}